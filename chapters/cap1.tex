\pagenumbering{arabic}

\chapter{Introdução}
\label{chp:introduction}

% \begin{quotation}[]{Poul Anderson}
% I have yet to see any problem, however complicated, which, when looked at in the
% right way, did not become still more complicated.
% \end{quotation}

\noindent Este capítulo contextualiza os principais assuntos abordados nesta dissertação, apresenta as motivações que levaram à escolha do tema, os objetivos gerais e específicos desta pesquisa, bem como a justificativa da investigação conduzida e suas principais contribuições.
\clearpage

% ----------------------------------------------------------
% seção
% ----------------------------------------------------------

\section{Contextualização}
% ----------------------------------------------------------

Desde a década de 1970, com a criação do modelo relacional por Edgar Frank Codd, a estrutura de armazenamento adotada por muitos desenvolvedores de sistemas da área de tecnologia da informação têm se baseado nesse conceito. A maioria dos sistemas gerenciadores de banco de dados (SGBD) que possui aceitação no mercado fazem uso desse modelo, por exemplo o MySQL, Oracle e Microsoft SQL Server. Porém, os requisitos de dados para o desenvolvimento de ferramentas de \textit{software} atual têm mudado significativamente, especialmente com o aumento das aplicações \textit{Web} (\cite{nasholm:2012}). Este segmento de aplicações exige alta escalabilidade e vazão, e SGBD relacionais muitas vezes não conseguem atender satisfatoriamente os seus requisitos de dados. Como alternativa a isso, novas abordagens de SGBD utilizando o termo de SGBD NoSQL tornaram-se popular (\cite{silva:2016}).

O termo NoSQL é constantemente interpretado como ''\emph{Not Only SQL}'' (não somente SQL), cujo SQL refere-se a linguagem disponibilizada pelos SGBD relacionais (\cite{nasholm:2012}). O propósito das abordagens NoSQL é oferecer alternativas onde os SGBD relacionais não apresentam um bom desempenho. Esse termo abrange diferentes tipos de sistemas. Em geral, SGBD NoSQL usam modelo de dados não-relacionais, com poucas definições de esquema, e geralmente, são executados em clusters. Alguns exemplos de SGBD NoSQL recentes são o Cassandra, MongoDB, Neo4J e o Riak (\cite{fowler:2013}).

Muitas empresas coletam e armazenam milhares de gigabytes de dados por dia, onde a análise desses dados representa uma vantagem competitiva no mercado. Por isso, há uma grande necessidade de novas arquiteturas para o suporte à decisão (\cite{liu:2013}). Para isso, uma das formas bastante utilizada é a criação de um ambiente \textit{data warehousing} responsável por providenciar informações estratégicas e esquematizadas a respeito do negócio (\cite{dayal:1997}).

Segundo a definição de \cite{kimball:2002}, \textit{data warehouse} (DW) é uma coleção de dados voltada para o processo de suporte à decisão, orientada por assunto, integrada, variante no tempo e não volátil. Os dados de diferentes fontes de dados são processados e integrados em um \textit{data warehouse} central através da Extração, Transformação e Carga (ETL) que é feita de maneira periódica. Os processos de ETL consistem em um conjunto de técnicas e ferramentas para transformar dados de múltiplas fontes de dados para fins de análise de negócio (\cite{silva:2016}). Ferramentas de ETL são sistemas de \textit{software} responsáveis por extrair dados de diversas fontes de dados, transformar, customizar e inseri-los no \textit{data warehouse}.

O projeto de ETL, ou seja, a criação dos seus processos, consome cerca de 70\% dos recursos de implantação de um DW, pois desenvolver esse projeto é crítico e custoso, tendo em vista que gerar dados incorretos pode acarretar em más decisões. Porém, por algum tempo pouca importância foi dada ao processo de ETL pelo fato de ser visto somente como uma atividade de suporte aos projetos de DW. Apenas a partir do ano 2000, a comunidade acadêmica passou a dar mais importância ao tema (\cite{silva:2012}). Atualmente, ainda existem dificuldades ao lidar com as soluções para ferramentas de ETL presentes na literatura. É comum que elas demonstrem mais importância aos SGBD relacionais. Pois, tradicionalmente o DW é implementado em um banco de dados relacional, onde o dado é armazenado nas tabelas fato e dimensões, que são descritas por um esquema em estrela (\cite{kimball:2002}). Por isso, para oferecer suporte aos sistemas que necessitem realizar os processos de ETL em banco de dados NoSQL, a proposta desse trabalho é especificar um \textit{framework} programável, flexível e integrado para modelagem e execução de processos de ETL em banco de dados NoSQL.

% ---
% Capitulo com exemplos de comandos inseridos de arquivo externo 
% ---
%\include{abntex2-modelo-include-comandos}
% ---

% ----------------------------------------------------------
% seção
% ----------------------------------------------------------
\section{Motivação}
% ----------------------------------------------------------

A integração de dados e os processos de ETL são procedimentos cruciais para a criação de \textit{data warehouses}. Porém, esses procedimentos são tradicionalmente desenvolvidos para dados em modelos relacionais, que representam apenas uma pequena parte dos dados mantidos por muitas empresas (\cite{darmont:2005}, \cite{russom:2007}, \cite{thomsen:2009}).

O uso generalizado da internet, web 2.0, redes sociais e sensores digitais produzem grandes volumes de dados. De fato, modelos de programação como o MapReduce (MR) introduzido pela Google (\cite{google:2004}), são executados continuamente para tratar mais de vinte Petabytes de dados por dia (\cite{dean:2008}). Esta explosão de dados é uma oportunidade para o surgimento de novas aplicações, como \textit{Big Data Analytics} (BDA); mas é, ao mesmo tempo, um problema dado as capacidades limitadas das máquinas e das aplicações tradicionais.

Esse grande volume de dados é conhecido como "\textit{Big Data}" e caracterizado pelos quatro "V": Volume, que implica a quantidade de dados que vão além das unidades de armazenamento usuais; a Velocidade com que esses dados são gerados e devem ser processados, a Variedade como diversidade de formatos e estruturas, e a Veracidade relacionada à precisão e confiabilidade dos dados. Assim, surgem novos paradigmas, tais como \textit{Cloud Computing} (Computação em Nuvem) e \textit{MapReduce} (MR), e novos modelos de dados são propostos para armazenamento de grandes volumes de dados, como o NoSQL (\textit{Not Only SQL}) (\cite{dean:2011}).

Dessa forma, existe uma demanda crescente para extrair, transformar e carregar grandes volumes de dados, apresentados de forma variada, em modelos não relacionais em um ambiente de suporte à decisão. Contudo, devido a complexidade desses dados, diferentes desafios estão surgindo quando lidamos com suas características, como por exemplo a heterogeneidade e distribuição desses dados, no ambiente de extração, transformação e carga de dados (\cite{salem:2012}).

Além disso, muitas empresas encontram dificuldades para lidar com as ferramentas de ETL disponíveis no mercado devido à sua longa curva de aprendizagem. Aprender a lidar com essas ferramentas pode ser muito custoso em termos financeiros e de tempo, e por isso, acabam optando desenvolver os seus processos por meio de uma linguagem de programação de propósito geral (\cite{awad:2011}, \cite{munoz:2009}).


%Uma maneira para organizar e manipular grandes volumes de dados sem utilizar um modelo relacional, e ainda processá-los e armazená-los de maneira distribuída é fazer o uso de BDs NoSQL (Bancos de Dados NoSQL) (\cite{scabora:2016}). Com isso, surge a necessidade de se promover meios para o uso desses bancos de dados em DWs. 

As pesquisas presentes na literatura sobre extração de dados em BDs NoSQL mostram que as ferramentas existentes no mercado propõem arquiteturas, metamodelos, aplicações e metodologias de modelagem para processos de ETL  ((\cite{silva:2016}, \cite{chevalier:2015}, \cite{liu:2013}).No entanto, elas não apresentam um framework programável em um ambiente integrado, e ainda ignoram conceitos importantes sobre frameworks, tais como reuso, flexibilidade, outros aspectos como a variedade dos dados e a falta de uma alternativa para substituir o SQL dos modelos relacionais.

Em geral, não é incomum observar que os especialistas em ETL utilizam interfaces textuais, enquanto que os não especialistas optem pelo uso de GUI. No entanto, mesmo que haja o envolvimento de membros não especializados em ETL, a implementação final dos processos é realizada por especialistas, os quais são mais eficientes quando utilizam interfaces de programação textuais (\cite{silva:2012}, \cite{mazanec:2012}).


Portanto, o aumento do uso de SGBD com modelos de dados não relacionais baseados no paradigma NoSQL e a falta de uma ferramenta programável, flexível e integrada, independente de plataforma que dê suporte à extração, transformação e carga para esses SGBD é a grande motivação deste trabalho.







% ----------------------------------------------------------
% Seção
% ----------------------------------------------------------
\section{Objetivos}

O objetivo principal desta pesquisa é especificar um \textit{framework} programável, flexível e integrado para modelagem e execução de processos de ETL de bancos de dados NoSQL. Baseamos nossa proposta nos princípios de flexibilidade, extensibilidade, reuso e inversão de controle, conforme recomenda a literatura sobre frameworks (\cite{awad:2011}, \cite{vassiliadis:2005}, \cite{fayad:1999}, \cite{fayad:1997}, \cite{darmont:2005}), além dos conceitos de desenvolvimento baseados em componentes apresentados na seção 2.1.3.

A arquitetura do ETL4NoSQL oferece uma interface de programação que contém elementos, tais como componentes de gerenciamento, leitura e escrita de dados, criação e execução de operações de ETL. O componente de gerenciamento é o responsável pelos fundamentos de ETL disponíveis na literatura (\cite{kimball:2004}). Ele possibilita as aplicações baseadas no ETL4NoSQL reutilizar os componentes para modelar seus processos de ETL. Além disso, a flexibilidade do framework proposto permite que sejam criados outros componentes que encapsulam regras de transformação para áreas especificas de ETL. Um componente especializado permite a construção de processos de ETL que não seriam possíveis ou que exigiriam o uso conjunto de muitos componentes genéricos do framework. Para possibilitar a criação de uma interface integrada para especialização e modelagem de processos de ETL, utilizamos o desenvolvimento baseado em componentes. Isto propiciou a implementação do ETL4NoSQL em uma linguagem de programação de propósito geral. Os objetivos específicos são detalhados a seguir.

\subsection{Objetivos Específicos}

Um dos objetivos específicos desta dissertação é apresentar a especificação e modelagem dos componentes do framework ETL4NoSQL, bem como suas interfaces e funcionalidades. Outro objetivo deste trabalho consiste em realizar um estudo experimental de \textit{software}, a fim de caracterizar as principais funcionalidades das ferramentas de ETL para BDs NoSQL. O estudo experimental tem como objetivo comparar o framework proposto neste trabalho e demonstrar suas vantagens e desvantagens em relação às ferramentas de ETL correlatas à este trabalho encontradas na literatura. Por fim, o nosso último objetivo é prover dois ambientes de ETL para facilitar a extração, transformação e carga de dados em DW modelados pelo esquema estrela, tendo em vista que este é o esquema de dados dimensional mais recomendado pela literatura (\cite{inmon:2002}, \cite{kimball:2002}). No primeiro ambiente utilizamos o SGBD MongoDB com dados sintéticos de ranking de restaurantes, e o segundo, o SGBD escolhido foi o CassandraDB com dados sintéticos de localizações de táxis.



%O primeiro objetivo específico desta pesquisa é estender a proposta do framework para facilitar a carga de dados de dois sistemas de BD NoSQL distintos baseado no mesmo paradigma NoSQL em um DW relacional modelados pelo esquema estrela, tendo em vista que este é o esquema de dados dimensionais mais recomendado pela literatura \cite{kimball:2002} \cite{Inmon: 2002}. O outro objetivo específico é ao invés de dar carga em um DW relacional fazer uso do mesmo sistema em um DW NoSQL, seguindo a metodologia adotada por \cite{chevalier:2015} em seu trabalho de pesquisa. Para isso, desenvolvemos dois frameworks especializados a partir do ETL4NoSQL em conformidade às peculiaridades dos processos de ETL nessas duas áreas de aplicação, os quais também são objetos de validação do ETL4NoSQL.

\section{Contribuições}

Uma das contribuições deste trabalho é o framework ETL4NoSQL. Ele permite extrair, transformar e carregar dados que estão armazenados em diversos SGBD NoSQL, ou até mesmo, repositórios de dados textuais e SGBD relacionais. A vantagem ao utilizar o ETL4NoSQL é sua natureza programável, flexível e integrada que facilita a modelagem e execução dos processos de ETL em banco de dados NoSQL.

Outra contribuição desta pesquisa é apresentar, por meio de um estudo experimental \textit{software} as principais funcionalidades de uma ferramenta de ETL, bem como possíveis melhorias, vantagens e desvantagens de acordo com os trabalhos correlatos encontrados na literatura.

Por fim, nossa última contribuição é oferecer duas aplicações de ETL, criadas a partir do ETL4NoSQL, utilizando domínios distintos baseados em dois sistemas NoSQL. 

\section{Organização do Trabalho}

Este trabalho está organizado de acordo com a seguinte estrutura:

\begin{itemize}
	\item \textbf{Fundamentação Teórica:} apresenta uma revisão de literatura sobre os principais assuntos abordados neste trabalho. São tratados temas a respeito de ETL, banco de dados NoSQL, frameworks, estudo experimental de \textit{software} e descreve os trabalhos correlatos encontrados na literatura a respeito de ferramentas de ETL.
	
	\item \textbf{O Framework ETL4NoSQL:} descreve os requisitos, a arquitetura de \textit{software} e implementação dos componentes do framework proposto.
			
	\item \textbf{Estudo Experimental de Software:} expõe o roteiro da experimentação de \textit{software} para ferramentas de ETL. Define o objetivo, o planejamento, a operação e o resultado do estudo.
	
	\item \textbf{Avaliação:} descreve aplicações de ETL4NoSQL para dois domínios de naturezas distintas, a fim de ilustrar a reusabilidade e flexibilidade do ETL4NoSQL, avaliando a proposta desta dissertação.
		
	\item \textbf{Considerações Finais:} expressa as limitações e ameaças à validade do trabalho, considerações finais e sugere de trabalhos futuros.	
	
\end{itemize}

