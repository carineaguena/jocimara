\chapter{Avaliação}

Este capítulo fornece aplicações exemplo baseadas no ETL4NoSQL em dois domínios de naturezas distintas utilizando SGBDs NoSQL para os dados de entrada. O desenvolvimento dessas aplicações pretende ilustrar a reusabilidade e flexibilidade do ETL4NoSQL em aplicações de ETL que apresentam requisitos distintos de inserção e transformação de dados, e assim avaliar o trabalho proposto neste documento.

Nas seções seguintes apresentaremos as aplicações utilizadas como exemplo para a avaliação do nosso trabalho. Demonstraremos a implementação de um processo de ETL para carga de dados em modelos multidimensionais utilizando os dados de entrada armazenados no SBGD MongoDB e outro no SGBD Cassandra.

\clearpage	

\section{Aplicações Baseadas no ETL4NoSQL}

Para avaliar o trabalho descrito neste documento, implementamos duas aplicações a partir do ETL4NoSQL. Elas são baseadas em dados sintéticos que podem ser encontrados em \cite{dataMongo} e \cite{dataCassandra}. 

Nossa motivação para escolha das duas aplicações foi para demonstrar a flexibilidade da ferramenta proposta ao lidar com SGBDs NoSQL de diferentes paradigmas, além dos SGBDs escolhidos serem bastante utilizados atualmente. Outra motivação foi facilitar a modelagem e carga desses dados no esquema estrela em DW.

Para a implementação, instanciamos a interface de leitura de dados (IDataMgr) e criamos as operações através da interface de operação (IOpMgr) de forma a modelar os dados, usando a interface de modelagem (IModelMgr), no modelo dimensional. Finalmente, pudemos processar os dados por meio da interface de processamento (IProcMgr) e carregá-los em um arquivo de saída com o formato JSON.

Dessa forma, o uso das interfaces do \textit{framework} ETL4NoSQL auxilia o projetista de ETL na modelagem dos processos a partir de seu reuso, além de permitir adequar os processos de acordo com o seu domínio por meio da flexibilização das interfaces dos componentes do ETL4NoSQL.

\section{Aplicação ETL4NoSQLMongoStar}

Para o desenvolvimento deste exemplo de aplicação utilizamos o SGBD Mongo e uma base de dados que armazena a avaliação dos clientes de vários restaurantes. A seguir apresentamos a estrutura da base de dados utilizada nesta aplicação.

\begin{lstlisting}[frame=single, language=Oberon-2, basicstyle=\tiny]

	{
		"address": {
			"building": "1007",
			"coord": [ -73.856077, 40.848447 ],
			"street": "Morris Park Ave",
			"zipcode": "10462"
		},
		"borough": "Bronx",
		"cuisine": "Bakery",
		"grades": [
			{ "date": { "$date": 1393804800000 }, "grade": "A", "score": 2 },
			{ "date": { "$date": 1378857600000 }, "grade": "A", "score": 6 },
			{ "date": { "$date": 1358985600000 }, "grade": "A", "score": 10 },
			{ "date": { "$date": 1322006400000 }, "grade": "A", "score": 9 },
			{ "date": { "$date": 1299715200000 }, "grade": "B", "score": 14 }
		],
		"name": "Morris Park Bake Shop",
		"restaurant_id": "30075445"
	}

\end{lstlisting}  

Dessa forma, na figura \ref{mongomultidim} definimos o modelo multidimensional para a esta estrutura de dados.

\begin{figure}[h!]
	\centering
	\includegraphics[scale=0.5]{fig/mongo_multidim.png}
	\caption{Modelo Multidimensional da aplicação ETL4NoSQLMongoStar}
	\label{mongomultidim}
\end{figure}

A partir do \textit{framework} ETL4NoSQL, pudemos estender uma nova aplicação denominada ETL4NoSQLMongoStar para desenvolver e executar os processos de ETL de forma a atender ao modelo multidimensional especificado para o exemplo utilizado.

\begin{figure}[h!]
	\centering
	\includegraphics[scale=0.4]{fig/ETL4NoSQLMongoStar.png}
	\caption{Tela da aplicação ETL4NoSQLMongoStar}
	\label{etl4nosqlmongostar}
\end{figure}

Na figura \ref{etl4nosqlmongostar} podemos ver a aplicação ETL4NoSQLMongoStar, no destaque 1 temos a árvore de arquivos da aplicação, incluindo os arquivos de saída, criados ao final da execução dos processos de ETL. Já no destaque 2 é importada as interfaces dos componentes do \textit{framework}. No destaque 3 é feita a leitura das bases de dados envolvidas nos processos, no destaque 4 é realizada a inserção das operações a serem realizadas e no destaque 5 os processos são executados. O destaque 6 apresenta as operações a serem realizadas pelo \textit{framework} para criar o modelo multidimensional do ETL4NoSQLMongoStar e sua saída de dados em arquivos JSON.

\section{Aplicação ETL4NoSQLCassandraStar}

Para a criação da aplicação exemplo ETL4NoSQLCassandraStar, cujo esta é uma aplicação estendida do \textit{framework} ETL4NoSQL proposta neste trabalho, utilizamos o SGBD Cassandra e a base de dados de localizações de táxis de acordo com sua latitude e longitude em um determinado momento. A estrutura de dados da base de dados de origem pode ser vista a seguir.

\begin{lstlisting}[frame=single, language=Oberon-2, basicstyle=\tiny]

	CREATE TABLE taxi.localizacoes (  
		taxi_id int, 
		date_time text,
		longitude text,
		latitude text 
	PRIMARY KEY (taxi_id));

\end{lstlisting}

Dessa forma, determinamos o modelo multidimensional para esta estrutura de dados apresentada na figura \ref{cassandramultidim}.

\begin{figure}[h!]
	\centering
	\includegraphics[scale=0.5]{fig/cassandra_multidim.png}
	\caption{Modelo Multidimensional da aplicação ETL4NoSQLCassandraStar}
	\label{cassandramultidim}
\end{figure}

Assim, utilizando a aplicação estendida do ETL4NoSQL, pudemos executar os processos de ETL para atender ao modelo multidimensional.

\begin{figure}[h!]
	\centering
	\includegraphics[scale=0.4]{fig/ETL4NoSQLCassandraStar.png}
	\caption{Tela da aplicação ETL4NoSQLCassandraStar}
	\label{ETL4NoSQLCassandraStar}
\end{figure}

Na figura \ref{ETL4NoSQLCassandraStar} podemos ver a aplicação ETL4NoSQLCassandraStar. Da mesma forma que a aplicação ETL4NoSQLMongoStar, no destaque 1 temos a árvore de arquivos da aplicação, incluindo os arquivos de saída, criados ao final da execução dos processos de ETL. Já no destaque 2 é importada as interfaces dos componentes do \textit{framework}. No destaque 3 é feita a leitura das bases de dados envolvidas nos processos, no destaque 4 é realizada a inserção das operações a serem realizadas e no destaque 5 os processos são executados. O destaque 6 apresenta as operações a serem realizadas pelo \textit{framework} para criar o modelo multidimensional do ETL4NoSQLCassandraStar e sua saída de dados em arquivos JSON.

\section{Considerações Finais}

Este capítulo apresentou duas aplicações de naturezas distintas estendidas do ETL4NoSQL, avaliando suas características de reusabilidade e flexibilidade. Uma aplicação foi baseada no SGBD NoSQL Mongo e a outra no SGBD NoSQL Cassandra. Utilizamos o ETL4NoSQLCassandraStar e o ETL4NoSQLMongoStar para desenvolver e executar os processos. Ao final da execução dos processos pudemos gerar um arquivo de saída em formato comum denominado JSON. 

No capítulo posterior, esta pesquisa é finalizada, expondo as principais contribuições, discussões sobre dificuldades e as ameaças do trabalho de pesquisa, os resultados e trabalhos futuros a partir do ETL4NoSQL.
