\chapter{Conclusão}


A existência de uma demanda crescente para integrar os dados modelados em vários tipos de estruturas em um repositório unificado, além de muitas empresas encontrarem dificuldades para lidar com as ferramentas ETL disponíveis no mercado motivaram esta pesquisa, cuja apresentou o ETL4NoSQL. Este capítulo resume o trabalho exposto nesta dissertação apresentando suas contribuições relevantes, seus desafios e dificuldades, seus resultados e trabalhos futuros a partir do ETL4NoSQL. 


Portanto, este trabalho propôs um \textit{framework} programável para desenvolvimento de aplicações de ETL que possibilita a extração, transformação e carga de dados armazenados em BDs NoSQL. O \textit{framework} possui um ambiente integrado para a leitura e escrita dos dados, além da modelagem e execução distribuída ou centralizada dos processos de ETL. O componente de gerenciamento de dados do \textit{framework} executa os processos de ETL em BDs NoSQL. Esse componente possibilita a leitura e manipulação de dados em BDs NoSQL, e também o armazenamento desses dados em bases deste tipo, oferecendo uma alternativa de modelo não relacional para a construção de DWs.

\clearpage

\section{Principais Contribuições}

Uma das principais contribuições do nosso trabalho é o \textit{framework} ETL4NoSQL, que engloba os requisitos de \textit{software} para ferramentas de ETL, a modelagem do domínio, através de seus três modelos: modelo conceitual, de casos de uso e de comportamento; a modelagem da especificação, da arquitetura e especificação dos componentes, identificação das interfaces de sistemas e de negócio, bem como a interação entre os componentes do ETL4NoSQL. O ETL4NoSQL pode ser utilizado para o desenvolvimento de novas aplicações de ETL. As especificações de seus componentes podem ser reutilizados para facilitar a especialização e instanciação de novas aplicações de ETL.

Outra contribuição deste trabalho é o estudo experimental de \textit{software} baseado nas ferramentas de ETL encontradas na literatura por esta pesquisa. O estudo teve como objetivo definir se o ETL4NoSQL é adequado para auxiliar no desenvolvimento de processos de ETL. Os resultados do estudo mostraram que o ETL4NoSQL possui um grau de similaridade de 70\% em relação com as outras 11 ferramentas estudadas nesta pesquisa, e que dessa similaridade 85,71\% das funcionalidades são consideradas úteis para desenvolver processos de ETL.

E por último, oferecemos duas aplicações de domínios distintos baseadas em SGBDs NoSQL, estendidas do ETL4NoSQL como forma de avaliar a flexibilidade e reusabilidade do \textit{framework} proposto neste trabalho. Os SGBDs utilizados para o desenvolvimento das aplicações foram o MongoDB e o Cassandra. Criamos o ETL4NoSQLCassandraStar e o ETL4NoSQLMongoStar para desenvolver e executar os processos, e ao final da execução dos processos pudemos gerar um arquivo de saída em formato comum, denominado JSON. Além disso, ressaltamos que para transformar os dados no esquema estrela utilizando as aplicações estendidas do ETL4NoSQL, bastou apenas adicionar os parâmetros de consulta de cada SGBD NoSQL para executar os processos de busca, junção e escrita, demonstrando que o \textit{framework} proposto é programável, no sentido de possibilitar a programação dos seus parâmetros; reusável, pois permite que seus componentes sejam reutilizados por várias aplicações; e finalmente, flexível, dado que é possível estendê-lo para atender vários domínios de aplicação. 

Dessa forma, podemos concluir que o objetivo da presente proposta, de especificar um \textit{framework} programável, flexível e integrado para extração, transformação e carga dos dados de bancos de dados NoSQL foi atingido de forma efetiva e satisfatória, no sentido de facilitar e flexibilizar a atividade de desenvolvimento de novas ferramentas de ETL.

\section{Discussão}

Na seção 2.2 apresentamos os trabalhos correlatos a esta pesquisa, nesta seção foi demonstrada as ferramentas de ETL mais citadas pela literatura. Podemos observar que a maioria das ferramentas pesquisadas foca no desempenho ao lidar com grandes volumes de dados e BDs NoSQL, porém nenhuma delas apresentou uma solução alternativa para modelagem de BDs NoSQL ficando a cargo do desenvolvedor encontrar meios para esquematizar os dados. Algumas delas, como a P-ETL, oferece alternativas para leituras de arquivos textuais, mas não dão enfoque aos BDs NoSQL.  Outras ferramentas como o ETLMR, CloudETL e BigETL utilizam processamento paralelo e distribuído para facilitar a execução de processos de ETL em grandes volumes de dados, contudo não lidam com a modelagem e leitura dos dados de SGBDs não relacionais.

Assim, como alternativa a isso, sugerimos o ETL4NoSQL que consiste em um \textit{framework} baseado em componentes, programável, flexível e integrado. Os seus componentes podem ser reutilizados, por meio de suas interfaces e especializados para atender as especificidades de cada domínio de aplicação.


\section{Trabalhos Futuros}

Como trabalhos futuros indicamos a execução de testes com dados reais. Adicionalmente, testar o desempenho com a execução dos processos de forma distribuída, centralizada e paralela. E finalmente, verificar a eficiência comparada aos outros métodos existentes para o desenvolvimento de aplicações de ETL.


