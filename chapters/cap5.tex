\chapter{Conclusão}


A existência de uma demanda crescente para integrar os dados estruturados, não estruturados e semi estruturados em um repositório unificado, além de muitas empresas encontrarem dificuldades para lidar com as ferramentas ETL disponíveis no mercado motivaram esta pesquisa, cuja apresentou o ETL4NoSQL. Este capítulo resume o trabalho apresentado nesta pesquisa de dissertação apresentando suas contribuições relevantes, seus desafios e dificuldades, seus resultados e trabalhos futuros a partir do ETL4NoSQL. 


Portanto, este trabalho propôs um \textit{framework} programável para desenvolvimento de sistemas de ETL que possibilita a integração de dados estruturados, não estruturados e semi estruturados armazenados em bases relacionais ou NoSQL. O \textit{framework} possui um ambiente integrado para a leitura e escrita dos dados, além da modelagem e execução distribuída ou centralizada dos processos de ETL. O componente de gerenciamento de dados do \textit{framework} integram dados estruturados, não estruturados e semi estruturados. Esse componente possibilita a leitura e manipulação de dados de bases NoSQL, e também o armazenamento desses dados em bases deste tipo, oferecendo uma alternativa não relacional para a construção de DWs.

\clearpage

\section{Principais Contribuições}

Uma das principais contribuições do nosso trabalho é o \textit{framework} ETL4NoSQL, que engloba os requisitos de software para ferramentas de ETL, a modelagem do domínio, através de seus três modelos: modelo conceitual, de casos de uso e de comportamento, a modelagem da especificação, determinando a arquitetura e especificando os componentes e identificando interfaces de sistemas e de negócio, bem como a interação entre os componentes do ETL4NoSQL. O ETL4NoSQL pode ser utilizado para o desenvolvimento de novas ferramentas de ETL para dados estruturados, semi estruturados e não estruturados, as especificações de seus componentes podem ser reutilizados para facilitar a especialização e instanciação.

Outra contribuição deste trabalho é o estudo experimental de software baseado nas ferramentas de ETL encontradas na literatura por esta pesquisa. O estudo teve como objetivo definir se o ETL4NoSQL é adequado para auxiliar no desenvolvimento de processos de ETL em dados estruturados, semi estruturados e não estruturados. Os resultados do estudo mostraram que o ETL4NoSQL possui um grau de similaridade de 70\% em relação com as outras 11 ferramentas estudadas nesta pesquisa, e que dessa similaridade 85,71\% das funcionalidades são consideradas úteis para desenvolver processos de ETL em dados estruturados, semi estruturados e não estruturados. 

Dessa forma, podemos concluir que o objetivo da presente proposta, de especificar um \textit{framework} programável, flexível e integrado para extração, transformação e carga dos dados em BDs NoSQL foi atingido de forma efetiva e satisfatória, no sentido de facilitar e flexibilizar a atividade de desenvolvimento de novas ferramentas de ETL para dados estruturados, semi estruturados e não estruturados.

\section{Discussão}

Na seção 2.2 apresentamos os trabalhos correlatos a esta pesquisa, nesta seção foi demonstrada as ferramentas de ETL mais citadas pela literatura. Podemos observar que a maioria das ferramentas pesquisadas foca no desempenho ao lidar com grandes volumes de dados e BDs NoSQL, porém nenhuma delas apresentou uma solução alternativa para modelagem de BDs NoSQL ficando a cargo do desenvolvedor encontrar meios para esquematizar os dados. Algumas delas, como a P-ETL, oferece alternativas para leituras de arquivos textuais, mas não dão enfoque aos BDs NoSQL.  Outras ferramentas como o ETLMR, CloudETL, BigETL utilizam processamento paralelo e distribuído para facilitar a execução de processos de ETL em grandes volumes de dados, contudo não lidam com a modelagem e leitura dos dados não relacionais.

Assim, como alternativa a isso, sugerimos o ETL4NoSQL que consiste em um \textit{framework} baseado em componentes, programável, flexível e integrado. Os seus componentes podem ser reutilizados, por meio de suas interfaces e especializados para atender as especificidades de cada demanda.


\section{Trabalhos Futuros}

Como trabalhos futuros indicamos a execução de testes com dados reais. Adicionalmente, testar o desempenho com a execução dos processos de forma distribuída, centralizada e paralela. E finalmente, verificar a eficiência comparada aos outros métodos existentes para o desenvolvimento de aplicações de ETL.


