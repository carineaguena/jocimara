
Os procedimentos cruciais para a criação de \textit{data warehouses} e sistemas \acp{bi} são a integração de dados e os processos de ETL. Porém, os processo para ETL e integração de dados são comumente desenvolvidos para dados estruturados em modelos relacionais que representam apenas uma pequena parte dos dados mantidos por muitas empresas. Por isso, existe uma demanda crescente para integrar também os dados não estruturados e semi estruturados em um repositório unificado. Porém, devido a complexidade desses dados, novos desafios estão surgindo ao lidarmos com a heterogeneidade e distribuição dos dados em um ambiente de integração de dados.

Ademais, muitas empresas encontram dificuldades ao lidar com as ferramentas ETL disponíveis no mercado. Aprender a lidar com essas ferramentas pode ser muito custoso em termos financeiros e de tempo, e por isso, acabam optando desenvolver os seus processos por meio de uma linguagem de programação de propósito geral.

Portanto, propomos um \textit{framework} programável para desenvolvimento de sistemas de ETL que possibilita a integração de dados estruturados, não estruturados e semi estruturados armazenados em bases relacionais ou NoSQL, denominado ETL4NoSQL. Apresentamos os componentes do \textit{framework} ETL4NoSQL, bem como suas funcionalidades. Além disso, realizamos um estudo experimental de software, cujo teve como objetivo definir se o ETL4NoSQL é adequado para auxiliar no desenvolvimento de processos de ETL em dados estruturados, semi estruturados e não estruturados. Os resultados do estudo mostraram que o ETL4NoSQL possui um grau de similaridade de 70\% em relação com as outras 11 ferramentas estudadas nesta pesquisa, e que dessa similaridade 85,71\% das funcionalidades são consideradas úteis para desenvolver processos de ETL em dados estruturados, semi estruturados e não estruturados. 

Assim, podemos concluir que o objetivo da presente proposta, de especificar um \textit{framework} programável, flexível e integrado para extração, transformação e carga dos dados em BDs NoSQL foi atingido de forma efetiva e satisfatória, no sentido de facilitar e flexibilizar a atividade de desenvolvimento de novas ferramentas de ETL para dados estruturados, semi estruturados e não estruturados.

\begin{keywords}
ETL, Frameworks, NoSQL, Data Warehouse, Estudo Experimental de Software
\end{keywords}