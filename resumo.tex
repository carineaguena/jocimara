
A integração de dados e os processos de extração, transformação e carga (ETL) são procedimentos cruciais para a criação de \textit{data warehouses} (DW). Porém, os processos de ETL e integração de dados são habitualmente desenvolvidos para dados estruturados em modelos relacionais (modelo ER), que representam apenas uma pequena parte dos dados mantidos por muitas empresas. Dessa forma, existe uma demanda crescente para extrair, transformar e carregar dados não estruturados em um repositório de dados unificado. Porém, devido a complexidade desses dados, novos desafios estão surgindo ao lidarmos com suas características, como por exemplo a heterogeneidade e distribuição, em um ambiente de extração, transformação e carga de dados.

Além disso, muitas empresas encontram dificuldades ao lidar com as ferramentas de ETL disponíveis no mercado. Aprender a lidar com essas ferramentas pode ser muito custoso em termos financeiros e de tempo, e por isso, acabam optando por desenvolver os seus processos utilizando uma linguagem de programação de propósito geral.

Portanto, neste trabalho propomos um \textit{framework} programável, flexível e integrado para auxiliar a modelagem e execução de processos de ETL, que possibilita a extração, transformação e carga de dados estruturados em esquemas de modelos de dados não relacionais, denominado ETL4NoSQL. Apresentamos os componentes do \textit{framework} ETL4NoSQL, bem como suas interfaces e funcionalidades. Ademais, realizamos um estudo experimental de \textit{software}, que teve como objetivo verificar se o ETL4NoSQL é adequado para auxiliar no desenvolvimento de processos de ETL. O estudo consistiu na análise das ferramentas de ETL encontradas na literatura, com o propósito de caracterizá-las por meio da intersecção de suas funcionalidades no contexto comparativo entre elas. Os resultados do estudo mostraram que o ETL4NoSQL possui um grau de similaridade, segundo o teste de associação Qui-quadrado, de 70\% em relação com as outras 11 ferramentas estudadas nesta pesquisa, e que desta similaridade 85,71\% das funcionalidades são consideradas fundamentais, de acordo com as características das ferramentas encontradas na literatura, para desenvolver processos de ETL. Finalmente, propomos um ambiente de implementação que permitiu reutilizar as interfaces de programação implementadas em dois tipos diferentes de SGBDs NoSQL.

Assim, podemos concluir que o objetivo da presente proposta, de especificar um \textit{framework} programável, flexível e integrado para extração, transformação e carga dos dados em BDs NoSQL foi atingido de forma efetiva e satisfatória, facilitando o reúso dos processos e flexibilizando as atividades de desenvolvimento de novos processos de ETL para BDs NoSQL.

\begin{keywords}
ETL, Frameworks, NoSQL, Data Warehouse, Estudo Experimental de Software
\end{keywords}