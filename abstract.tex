The crucial procedures for creating data warehouses and BI systems are data integration and ETL processes. However, processes for ETL and data integration are commonly developed for data structured in relational models that represent only a small part of the data maintained by many companies. Therefore, there is a growing demand to integrate both unstructured and semi-structured data into a unified repository. However, due to the complexity of these data, new challenges are emerging as we deal with the heterogeneity and distribution of data in the integration environment.

In addition, many companies encounter difficulties in dealing with the ETL tools available in the market. Learning how to handle these tools can be very costly in terms of time and financially, and so they end up opting to develop their processes through a general purpose programming language.

Therefore, we propose a programmable framework for the development of ETL systems that allows the integration of structured, unstructured and semi-structured data stored in relational bases or NoSQL, called ETL4NoSQL. We present the components of the ETL4NoSQL framework, as well as its functionalities. In addition, we conducted an experimental software study, whose purpose was to determine if ETL4NoSQL is adequate to assist in the development of ETL processes in structured, semi-structured and unstructured data. The results of the study showed that ETL4NoSQL has a degree of similarity of 70\% in relation to the other 11 tools studied in this research, and that of this similarity 85.71\% of the functionalities are considered useful to develop ETL processes in structured, semi structured and unstructured data.

Thus, we can conclude that the objective of the present proposal to specify a programmable, flexible and integrated framework for extracting, transforming and loading data into NoSQL BDs has been achieved in an effective and satisfactory way, in order to facilitate and make flexible the development activity of new ETL tools for structured, semi-structured and unstructured data.


\begin{keywords}
	ETL, Frameworks, NoSQL, Data Warehouse, Experimental Software Study
\end{keywords}
