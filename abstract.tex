Data integration and data extraction, transformation, and data loading (ETL) processes are crucial procedures for the creation of data warehouses (DW). However, the ETL and data integration processes are usually developed for data structured in relational models (ER model), which represent only a small part of the data maintained by many companies. Thus, there is a growing demand to extract, transform, and load structured data into non-relational data models into a unified data repository. However, due to the complexity of these data models, new challenges are emerging when we deal with their characteristics, such as heterogeneity and data distribution, in an environment of data extraction, transformation and data loading.

In addition, many companies encounter difficulties in dealing with the ETL tools available in the market. Learning how to handle these tools can be very costly in terms of time and money, and so they end up opting to develop their processes using a general purpose programming language.

Therefore, in this work we propose a programmable, flexible and integrated framework to support the modeling and execution of ETL processes, which enables the extraction, transformation and loading of structured data in nonrelational data models, called ETL4NoSQL. We present the components of the ETL4NoSQL framework, as well as its interfaces and functionalities. In addition, we conducted an experimental software study, which aimed to verify if ETL4NoSQL is suitable to assist in the development of ETL processes. The study consisted of the analysis of the ETL tools found in the literature, with the purpose of characterizing them through the intersection of their functionalities in the comparative context between them. The results of the study showed that ETL4NoSQL has a degree of similarity, according to the test Chi-square association, of 70\% in relation to the other 11 tools studied in this research, and that of this similarity, 85.71\% of the functionalities are considered fundamental , according to the characteristics of the tools found in the literature, to develop ETL processes. Finally, we propose an implementation environment that allowed reusing and extending the programming interfaces of ETL4NoSQL to develop ETL applications using two different types of NoSQL DBMS.

Thus, we can conclude that the objective of this proposal to specify a programmable, flexible and integrated framework for extracting, transforming and loading the data for NoSQL DBs has been achieved in an effective and satisfactory way, facilitating the reuse of processes and making flexible the development activities of new ETL applications in NoSQL DBs.

\begin{keywords}
	ETL, Frameworks, NoSQL, Data Warehouse, Experimental Software Study
\end{keywords}
